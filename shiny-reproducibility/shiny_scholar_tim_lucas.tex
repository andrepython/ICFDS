\documentclass[handout, aspectratio = 169, xcolor={dvipsnames}]{beamer}

\usetheme{Pittsburgh}

\usepackage{enumitem, colortbl}
\usepackage[pagewise, mathlines]{lineno}
\usepackage{comment}
\usepackage{listings}
\usepackage{fancyvrb}
\usepackage{tabularx}
\usepackage{array}
\usepackage{colortbl}

\setbeamercovered{invisible}


\makeatletter
\def\blfootnote{\gdef\@thefnmark{}\@footnotetext}
\makeatother




\usepackage[dvipsnames]{xcolor}


\graphicspath{{figures/}}
%\usepackage{tikz}

%\usetikzlibrary{automata,positioning}

%\usepackage{animate}
%\usepackage[scaled]{berasans}
\renewcommand*\familydefault{\sfdefault} %% Only if the base font of the document is to be sans serif

\usepackage[T1]{fontenc}

%\usepackage{epstopdf}
\setbeamertemplate{navigation symbols}{}
%\setbeamertemplate{background}{\includegraphics[width=\paperwidth]{../../Brown-Paper-6.jpg}}

\title % (optional, only for long titles)
{}
\author % (optional, for multiple authors)
{Tim~\small{C.D.}~\normalsize{Lucas}}
\institute % (optional)
{
}
%\usepackage{PSTricks}

\makeatletter
\newlength\beamerleftmargin
\setlength\beamerleftmargin{\Gm@lmargin}


\begin{document}

\begin{frame}

\frametitle{\{shinyscholar\} for modular, reproducible, maintainable \{shiny\} apps.}

\framesubtitle{{\color{black} \textbf{Tim C.D. Lucas}, Simon Smart\\
\hbox{}\\
{\small Dept of Popn Health Sciences, University of Leicester}}

\vspace{0.5cm}


\includegraphics[height=7pt]{../../bsky.jpeg}{\color{black}  \hspace{0.6mm}\footnotesize{@}\small{StatsForBios.bsky.social}}
\vspace{0.5cm}

\includegraphics[height=7pt]{../../Twitter_logo_blue-small.png}{\color{black} \hspace{0.6mm}\footnotesize{@}\small{timcdlucas}  \footnotesize{@}\small{statsforbios}}\\

\includegraphics[height=7pt]{../../Ar_Icon_Contact.pdf}{\color{black}  \hspace{0.6mm}\small{tim.lucas}\footnotesize{@}\small{le.ac.uk}}




}

\vspace{-3cm}
\hspace{-1cm}
\includegraphics[trim = {0 2cm 0 0}, clip, height=0.57\textheight]{app_screenshots_for_tim6.png}


\end{frame}




% 15 minutes?
\begin{frame}
\frametitle{shinyscholar (Simon Smart)}
www.github.com/simon-smart88/shinyscholar


\end{frame} 





% todo add github link!

% 15 minutes?
\begin{frame}
\frametitle{Outline}

\begin{itemize}
\item The problem
\item The solution \{shinyscholar\}
\item Quick guide
\item NOT a full workshop tutorial
\end{itemize}

\end{frame} 




% 15 minutes?
\begin{frame}
\frametitle{The problem:}

\begin{itemize}
\item 1000s of lines in one script.
\item No unit tests.
\item Unreproducible.
\item Hard to maintain and extend.
\item Not modular.
\end{itemize}

\end{frame} 








\begin{frame}
\frametitle{\{Wallace\}}
\includegraphics[width=0.95\textwidth]{wallace_prediction.png}

\end{frame} 


\begin{frame}
\frametitle{\{Wallace\}: reproducibility}
\includegraphics[width=0.95\textwidth]{wallace_code.png}

\end{frame} 


\begin{frame}
\frametitle{\{disagapp\}: what we've done}
\includegraphics[height=1.3\textheight]{drawing-2.pdf}

\end{frame} 




\begin{frame}
\frametitle{\{shinyscholar\} (Simon Smart)}
\includegraphics[width=0.95\textwidth]{app_screenshots_for_tim7.png}
\end{frame} 




\begin{frame}
\frametitle{Quick guide}
\framesubtitle{Five (six) Ps}

Proper Planning Prevents (Piss) Poor Performance


\end{frame} 


\begin{frame}
\frametitle{Components}
\includegraphics[width=0.95\textwidth]{schematix.pdf}
\end{frame} 

\begin{frame}
\frametitle{Modules}
\includegraphics[width=0.95\textwidth]{schematix2.pdf}
\end{frame} 



\begin{frame}[fragile]
\frametitle{\{shinyscholar\}}
\footnotesize{
\vspace{1mm}
\begin{Verbatim}
modules <- data.frame(
  "component" = c("load", "load", "plot", "plot"),
  "long_component" = c("Load data", "Load data", "Plot data", "Plot data"),
  "module" = c("user", "database", "histogram", "scatter"),
  "long_module" = c("Upload your own data", "Query a database to obtain data",
    "Plot the data as a histogram", "Plot the data as a scatterplot"),
  "map" = c(FALSE, FALSE, FALSE, FALSE),
  "result" = c(FALSE, FALSE, TRUE, TRUE),
  "rmd" = c(TRUE, TRUE, TRUE, TRUE),
  "save" = c(TRUE, TRUE, TRUE, TRUE))

common_objects = c("data", "histogram", "scatter")

\end{Verbatim}
}
\end{frame} 




\begin{frame}[fragile]
\frametitle{\{shinyscholar\}}
\footnotesize{
\vspace{1mm}



\renewcommand{\FancyVerbFormatLine}[1]{%
   \ifnum\value{FancyVerbLine}=2\color{cyan}#1%
   \else #1\fi}

\begin{Verbatim}
modules <- data.frame(
  "component" = c("load", "load", "plot", "plot"),
  "long_component" = c("Load data", "Load data", "Plot data", "Plot data"),
  "module" = c("user", "database", "histogram", "scatter"),
  "long_module" = c("Upload your own data", "Query a database to obtain data",
    "Plot the data as a histogram", "Plot the data as a scatterplot"),
  "map" = c(FALSE, FALSE, FALSE, FALSE),
  "result" = c(FALSE, FALSE, TRUE, TRUE),
  "rmd" = c(TRUE, TRUE, TRUE, TRUE),
  "save" = c(TRUE, TRUE, TRUE, TRUE))

common_objects = c("data", "histogram", "scatter")

\end{Verbatim}
}
\end{frame} 





\begin{frame}[fragile]
\frametitle{\{shinyscholar\}}
\footnotesize{
\vspace{1mm}



\renewcommand{\FancyVerbFormatLine}[1]{%
   \ifnum\value{FancyVerbLine}=4\color{cyan}#1%
   \else #1\fi}

\begin{Verbatim}
modules <- data.frame(
  "component" = c("load", "load", "plot", "plot"),
  "long_component" = c("Load data", "Load data", "Plot data", "Plot data"),
  "module" = c("user", "database", "histogram", "scatter"),
  "long_module" = c("Upload your own data", "Query a database to obtain data",
    "Plot the data as a histogram", "Plot the data as a scatterplot"),
  "map" = c(FALSE, FALSE, FALSE, FALSE),
  "result" = c(FALSE, FALSE, TRUE, TRUE),
  "rmd" = c(TRUE, TRUE, TRUE, TRUE),
  "save" = c(TRUE, TRUE, TRUE, TRUE))

common_objects = c("data", "histogram", "scatter")

\end{Verbatim}
}
\end{frame} 


\begin{frame}[fragile]
\frametitle{\{shinyscholar\}.}
\footnotesize{
\vspace{1mm}
\begin{Verbatim}

shinyscholar::create_template(path = ".", name = "demo", 
                              author = "Simon E. H. Smart",
                              include_map = FALSE, 
                              include_table = TRUE, 
                              include_code = TRUE, 
                              common_objects = common_objects, 
                              modules = modules, install = TRUE)


\end{Verbatim}
}
\end{frame} 


\begin{frame}
\frametitle{\{shinyscholar\} (Simon Smart)}
\includegraphics[width=0.95\textwidth]{shinyscholar_new_app_dir.PNG}
\end{frame} 


\begin{frame}
\frametitle{\{shinyscholar\} (Simon Smart)}
\includegraphics[width=0.95\textwidth]{shinyscholar_new_app_shinydir.png}
\end{frame} 




\begin{frame}[fragile]
\frametitle{\{shinyscholar\}.}
\footnotesize{
\vspace{1mm}
\begin{Verbatim}
# We created a pkg called {demo}
library(demo)

# It has a shiny app called demo
run_demo()
\end{Verbatim}
}
\end{frame} 


\begin{frame}
\frametitle{\{shinyscholar\} (Simon Smart)}
\includegraphics[width=0.95\textwidth]{shinyscholar_new_app_intro.png}
\end{frame} 

\begin{frame}
\frametitle{\{shinyscholar\} (Simon Smart)}
\includegraphics[width=0.95\textwidth]{shinyscholar_new_app_components.png}
\end{frame} 






\begin{frame}
\frametitle{\{shinyscholar\} (Simon Smart)}
\includegraphics[width=0.95\textwidth]{session_code.png}
\end{frame} 




\begin{frame}
\frametitle{Modules}
Each module calls \emph{one} function in the top R package.
\end{frame} 

% so you can test
% split shiny from non-shiny
% and for reproducibility (this is what is in markdown).


\begin{frame}
\frametitle{\{shinyscholar\} (Simon Smart)}
\includegraphics[width=0.95\textwidth]{shinyscholar_new_app_R.png}
\end{frame} 





\begin{frame}
\frametitle{To fully edit load database module.}

\begin{itemize}
\item Add code to \texttt{load\_database.R} in package.
\item 
\item Add code to \texttt{load\_database.R} in shiny app.
\item Edit \texttt{load\_database.md}
\item Edit \texttt{load\_database.Rmd}
\item Add packages to \texttt{.yml}
\end{itemize}

\end{frame} 




\begin{frame}[fragile]
\frametitle{Add code to \texttt{load\_database.R} in package}
\footnotesize{
\vspace{1mm}
\begin{Verbatim}
load_user <- function(x){
  return(NULL)
}

\end{Verbatim}
}
\end{frame} 


\begin{frame}[fragile]
\frametitle{Add code to \texttt{load\_database.R} in package}
\footnotesize{
\vspace{1mm}
\begin{Verbatim}
load_user <- function(x){
  
  data <- data.frame(x = rnorm(100),
                     y = rnorm(100))
  
  return(data)
}

\end{Verbatim}
}
\end{frame} 



\begin{frame}
\frametitle{Why put code in a package?}

\begin{itemize}
\item Testable
\item Transportable
\item Used in reproducible Rmarkdown
\end{itemize}

\end{frame} 


\begin{frame}[fragile]
\frametitle{Add code to \texttt{load\_user.R} in shiny app}
\footnotesize{
\vspace{1mm}
\begin{Verbatim}

load_user_module_server <- function(id, common, parent_session) {
  moduleServer(id, function(input, output, session) {

  observeEvent(input$run, {
    # WARNING ####

    # FUNCTION CALL ####
    
    
    # LOAD INTO COMMON ####

\end{Verbatim}
}
\end{frame} 


\begin{frame}[fragile]
\frametitle{Add code to \texttt{load\_user.R} in shiny app}
\footnotesize{
\vspace{1mm}
\begin{Verbatim}

load_user_module_server <- function(id, common, parent_session) {
  moduleServer(id, function(input, output, session) {

  observeEvent(input$run, {
    # WARNING ####

    # FUNCTION CALL ####
    
    data <- load_user()
    common$data <- data
    
    # LOAD INTO COMMON ####

\end{Verbatim}
}
\end{frame} 





\begin{frame}
\frametitle{Help}

\begin{itemize}
\item \{shinyscholar\} is itself a \{shinyscholar\} package/app, so can be used as a guide.
\item Vignette and README.
\item Paper (coming soon)
\item Email us! Happy to help.
\end{itemize}

\end{frame} 











\begin{frame}
\frametitle{\{disagapp\} (Simon Smart)}
\includegraphics[width=0.95\textwidth]{app_screenshots_for_tim2.png}
\end{frame} 








    
\begin{frame}

\frametitle{Please ask me some questions}


\includegraphics[height=7pt]{../../bsky.jpeg}{\color{black}  \hspace{0.6mm}\footnotesize{@}\small{StatsForBios.bsky.social}}
\vspace{0.5cm}

\includegraphics[height=7pt]{../../Twitter_logo_blue-small.png}{\color{black} \hspace{0.6mm}\footnotesize{@}\small{timcdlucas}  \footnotesize{@}\small{statsforbios}}\\

\includegraphics[height=7pt]{../../Ar_Icon_Contact.pdf}{\color{black}  \hspace{0.6mm}\small{tim.lucas}\footnotesize{@}\small{le.ac.uk}}


\includegraphics[height=7pt]{../../Ar_Icon_Contact.pdf}{\color{black}  \hspace{0.6mm}\small{ss1545}\footnotesize{@}\small{le.ac.uk}}

\vspace{2cm}

www.github.com/simon-smart88/shinyscholar


\hfill % pushes logo the right

\end{frame}







%thanks


\end{document}
